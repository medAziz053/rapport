\setcounter{mtc}{5} %indique le num�ro r�el du chapitre, pour la mini table des mati�res
\chapter{�tude Th�orique}
\minitoc  %insert la minitoc

\graphicspath{{Chapitre1/figures/}}
%==============================================================================
\pagestyle{fancy}
\fancyhf{}
\fancyhead[R]{\bfseries\rightmark}
\fancyfoot[R]{\thepage}
\renewcommand{\headrulewidth}{0.5pt}
\renewcommand{\footrulewidth}{0pt}
\renewcommand{\chaptermark}[1]{\markboth{\MakeUppercase{\chaptername~\thechapter. #1 }}{}}
\renewcommand{\sectionmark}[1]{\markright{\thechapter.\thesection~ #1}}

\begin{spacing}{1.2}
%==============================================================================

\section*{Introduction}
Une �tude th�orique \cite{YOUSFI2015} peut contenir l'une et/ou l'autre de ces deux parties :
\section{�tat de l'art} 
C'est une �tude assez d�taill�e sur ce qui existe sur le march� ou dans la litt�rature (d'o� 
le terme �tat de l'art), qui permet de r�pondre � la probl�matique. L'id�e ici est de faire 
un comparatif entre les solutions existantes, mais surtout d'analyser le r�sultat de cette 
comparaison et de dire pourquoi ne sont-elles pas satisfaisantes pour r�pondre � votre 
probl�matique.
\begin{figure}[!ht]\centering
\includegraphics[scale=0.9]{art.jpg}
\caption{�tat de l'art}
\label{fig:fig1}
\end{figure}
\section{�tude de l'existant}
Elle est en g�n�ral r�alis�e quand on va d�velopper un module suppl�mentaire sur un 
logiciel existant, ou si on va modifier une application existante. L'�tude de l'existant
consiste � expliquer ce qui existe d�j� dans votre environnement de travail.

\section*{Conclusion}
La conclusion est en g�n�ral sans num�rotation, et n'appara�t pas dans la table des mati�res.


%==============================================================================
\end{spacing}
