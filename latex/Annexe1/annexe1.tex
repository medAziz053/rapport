\setcounter{figure}{0} 
\setcounter{table}{0}
\setcounter{footnote}{0}
\setcounter{equation}{0}
\pagestyle{fancy}
\fancyhf{}
\renewcommand{\chaptermark}[1]{\markboth{\MakeUppercase{#1 }}{}}
\renewcommand{\sectionmark}[1]{\markright{\thesection~ #1}}
\fancyhead[RO]{\bfseries\rightmark}
\fancyhead[LE]{\bfseries\leftmark}
\fancyfoot[RO]{\thepage}
\fancyfoot[LE]{\thepage}
\renewcommand{\headrulewidth}{0.5pt}
\renewcommand{\footrulewidth}{0pt}

\makeatletter
\renewcommand\thefigure{A.\arabic{figure}}
\renewcommand\thetable{A.\arabic{table}} 
\makeatother

\chapter{Annexe : Remarques Diverses}
\graphicspath{{Annexe1/figures/}}
%==========================================================================

%    Annexe

%===========================================================================
\begin{itemize}
\item Un rapport doit toujours �tre bien num�rot�;
\item De pr�f�rence, ne pas utiliser plus que deux couleurs, ni un caract�re fantaisiste; 
\item Essayer de toujours garder votre rapport sobre et professionnel; 
\item Ne jamais utiliser de je ni de on, mais toujours le nous (m�me si tu as tout fait tout seul); 
\item Si on n'a pas de paragraphe 1.2, ne pas mettre de 1.1;
\item TOUJOURS, TOUJOURS faire relire votre rapport � quelqu'un d'autre (de pr�f�rence qui n'est pas du domaine) pour vous corriger les fautes d'orthographe et de fran�ais;
\item Toujours valoriser votre travail : votre contribution doit �tre bien claire et mise en �vidence; 
\item Dans chaque chapitre, on doit trouver une introduction et une conclusion;
\item Ayez toujours un fil conducteur dans votre rapport. Il faut que le lecteur suive un raisonnement bien clair, et trouve la relation entre les diff�rentes parties;
\item Il faut toujours que les abr�viations soient d�finies au moins la premi�re fois o� elles sont utilis�es. Si vous en avez beaucoup, utilisez un glossaire.
\item Vous avez tendance, en d�crivant  l'environnement mat�riel, � parler de votre ordinateur, sur lequel vous avez d�velopp� : ceci est inutile. Dans cette partie, on ne cite que le mat�riel qui a une influence sur votre application. Que vous l'ayez d�velopp� sur Windows Vista ou sur Ubuntu n'a aucune importance;
\item Ne jamais mettre de titres en fin de page; 
\item Essayer toujours d'utiliser des termes fran�ais, et �viter l'anglicisme. Si certains termes  sont plus connus en  anglais, donner leur �quivalent en fran�ais la premi�re fois que vous les utilisez, puis utilisez le mot anglais, mais en italique;
\item �viter les phrases trop longues : clair et concis, c'est la r�gle g�n�rale !\\

\newpage

\textbf{Rappelez vous que votre rapport est le visage de votre travail : un mauvais rapport peut �clipser de l'excellent travail. Alors pr�tez-y l'attention n�cessaire.}

 
\begin{figure}[!ht]\centering
\includegraphics[scale=0.5]{ingenieur.jpg}
\end{figure}
\end{itemize}

